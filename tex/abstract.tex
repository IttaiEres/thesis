\abstract
A primary goal in human genetics is to understand how genetic variation affects phenotypic variation observed between different individuals.  Elucidating these connections is crucial to understanding the molecular mechanisms and causes that lead to differences in traits and diseases among humans, and enables the discovery of therapeutic interventions that can improve human health. Many of the genetic variants associated with trait variation thus far are in non-coding regions of the genome, emphasizing a need to characterize non-coding sequence elements through functional genomics approaches. As opposed to variation in protein-coding sequences, genetic variation at non-coding loci has not been decoded, impeding a rapid understanding of its downstream functional effects. What is clear is that many non-coding loci are likely implicated in gene regulation, and understanding precisely what they are and how they act will require application of a wide variety of functional genomics techniques. In particular, one emerging technique assesses 3D genome structure, which can help connect regulatory loci to the genes they affect, and is rapidly expanding our understanding of gene regulatory networks and the noncoding genome. Various epigenetic features, such as 3D genome structure, can now be compared genome-wide across humans and non-human primates, broadening our understanding of evolution and gene regulation. My thesis work uses these paradigms to understand the interplay between 3D chromatin organization and gene expression across evolution. In Chapter 2, I apply RNA sequencing and chromosome conformation capture sequencing to human and chimpanzee induced pluripotent stem cells, in order to understand how divergence in 3D regulatory landscapes across these species affects expression divergence. As expected, this work demonstrates that reorganization of 3D genome structure contributes to gene regulatory evolution in primates. In Chapter 3, I critically assess existing evidence from other studies that have led many to conclude there is high evolutionary conservation of topologically associating domains (TADs, a large-scale feature of 3D genome organization). A thorough examination of the available data suggests such a conclusion may be unwarranted. Finally, in Chapter 4, I summarize the insights gained from this work, assess the state of the 3D genome field in general, and suggest next steps for future research.