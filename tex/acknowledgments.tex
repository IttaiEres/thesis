\acknowledgments

I would like to express extreme gratitude to all the people who supported, guided, mentored, and assisted me before and throughout my PhD. None of this work would have been possible without the mental, emotional, and fiscal support of numerous other trainees and faculty.

I am very thankful to my advisor, Yoav Gilad. Our interactions and conversations throughout the course of the PhD have not only significantly shaped my scientific thinking, but have also made my grad school experience an enjoyable and educational one. I found Yoav as an advisor to be a very understanding and compassionate, two qualities that made for a very enriching training experience. I am especially grateful that he encouraged me to pursue comparative primate genomics through the lens of my own interest in 3D genomics, while also always keeping me grounded in pursuing tangible results and publication. Additionally, Yoav was flexible and versatile when it was most needed (i.e. during the COVID-19 pandemic), and helped ensure I was still able to do quality scientific work and graduate. In so many ways, I could not have asked for a better PhD advisor.

I am also thankful to my thesis committee members, Marcelo Nobrega, John Novembre, and Xin He. Their guidance and insight throughout our committee meetings helped steer me in the right direction to make interesting inferences from my data, and, ultimately, to graduate. I also greatly enjoyed our "big picture" discussions at these meetings, that helped contextualize my work in the broader space of epigenetics, evolution, and 3D genome structure. Additionally, I am thankful to Matthew Stephens for his statistical advice and explanations at different times throughout my PhD.

I would like to thank my collaborators, Kevin Luo, Lauren Blake, and Joyce Hsiao. Kevin was instrumental in helping direct my early analyses of Hi-C data, especially with respect to quality control metrics. Lauren and Joyce were both extraordinarily helpful for properly carrying out the mediation analysis assessing the effect of 3D chromatin structure on gene expression differences.

In general, I would like to thank all members of the Gilad lab. My time in the lab was an extremely enjoyable and educational experience. I would like to give special thanks to Bryan Pavlovic for mentoring me early on in my PhD studies. Numerous other lab members provided helpful scientific guidance and were great friends to me during my time in the lab; in particular I would like to thank Jonathan Burnett, Nick Banovich, Seb Pott, Genevieve Housman, Kenneth Barr, Benjamin Fair, Ben Umans, Reem Elorbany, Katie Rhodes, Wenhe Lin, Erik McIntire, Sidney Wang, Courtney Burrows, Po-Yuan Tung, John Blischak, Briana Mittleman, Natalia Gonzales, Irene Gallego Romero, Amy Mitrano, Emilie Briscoe, Stephanie Lozano, Marsha Myrthil, and Claudia Chavarria. I am also grateful to other members of the broader human genetics community during my time at University of Chicago:  Abhishek Sarkar, Joe Marcus, Arjun Biddanda, Nick Knoblauch, Kevin Magnaye, Sarah Urbut, Oni Basu, Vincent Lynch, and countless other trainees, faculty, and staff.

A wide variety of different administrators provided me with unparalleled support during my PhD. I am especially thankful to Sue Levison, who has always gone above and beyond the call of duty to make me and other students feel welcome, and often helped assuage my worries. I would also like to thank Diane Hall, Melissa Lindberg, Candice Lewis, Anita Williams, and Carolyn Brown. In the first few years of my PhD, I was lucky to be funded by the Genetics and Regulation Training Grant (T32GM007197) from the National Institutes of Health. Many thanks to the NIH and to Lucia Rothman-Denes, the grant manager.

I would like to thank my various friends throughout grad school who provided many different kinds of support: Andrew Tremain, Amelia Joslin, UnJin Lee, Frances Lee, Charlie Lang, Erin Fry, Katie Mika, Aarti Venkat, Ryan Duncombe, Chris Stamper, Jess Fessler, Andr�s Moya-Rodriguez, Keelan Armstrong, Ben Ward, Addison Hughes, Michael Hustedde, Drew Waford, Asher Mayerson, Alex Advani, Toufic Mayassi, Sangman Kim, Jason Lui, Logan Poole, Charlie Dulberger, and countless others. I am also grateful for the different groups that I played both IM and pick-up soccer with throughout grad school. Finally, I would like to thank my partner, Hannah Morley, for the many different ways in which she has supported me throughout this process.

Lastly and most importantly, I would like to thank my family. My parents, Avichai and Ronit, supported me in so many different ways growing up, and gave me the educational and experiential opportunities that enabled me to pursue a PhD to begin with. I could not be more grateful for all their support throughout my life. My siblings, Tomer and Merav, have also been instrumental to my development both emotionally and intellectually, and I am thankful for the different ways in which they have supported and lifted me up. None of what I have accomplished would have been possible without the amazing support my whole family has continuously provided, and I am forever indebted to them.