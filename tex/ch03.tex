\chapter{A TAD Skeptic: Is 3D Genome Topology Evolutionarily Conserved?}\label{ch:ch03}

\section[Abstract]{Abstract\footnotemark}

The notion that topologically associating domains (TADs) are highly conserved across species has practically become an axiom in the field of 3D genome research. But what exactly do we mean by `highly conserved', and what are the actual comparative data that support this notion? To address these questions, we performed a historical review of the relevant literature, and retraced numerous citation chains to reveal the primary data that were used as the basis for the widely accepted conclusion that TADs are highly conserved across evolution. A thorough review of the available evidence suggests the answer may be more complex than what is commonly presented.

\footnotetext{Citation for chapter: Eres, Ittai E, and Gilad, Y. A TAD Skeptic: Is 3D Genome Topology Evolutionarily Conserved? \textit{Manuscript in review at Trends in Genetics}}

\section{What are TADs?}
Some of the most fascinating features to have emerged from research into 3D genome conformation are topologically associating domains (TADs). Originally discovered through the analysis of Hi-C data \cite{Dixon.2012, Nora.2012, Hou.2012, Sexton.2012}, TADs appear on a chromatin contact map as large squares of enhanced contact frequency rising off the diagonal. The original method for algorithmic TAD inference used a `directionality index' \cite{Dixon.2012} to define TADs as segments of the genome that are more connected to each other than to other regions.  In turn, TAD boundaries were defined as segments of the genome that are characterized by a sharp transition between upstream and downstream highly connected regions. Originally applying directionality index to Hi-C data at the relative low resolution of 40 kb, early studies reported that TADs are non-overlapping, highly self-interacting megabase-scale structures. More recent studies, using higher-resolution contact maps, as well as different inference algorithms, have revealed TAD structures at much smaller scales, and often nested within each other \cite{Phillips-Cremins.2013, Berlivet.2013, Rao.2014}. The precise nature of these features is still a matter of debate, with various definitions of TADs shifting as new algorithms arise and new discoveries are made about the mechanisms behind TAD formation (e.g. loop extrusion and compartmentalization) \cite{Phillips-Cremins.2013, Rao.2014, Filippova.2014, Dixon.2016, Zhan.2017, Nuebler.2018, Fudenberg.2016, Sanborn.2015}. Previous studies have found relatively low concordance of TADs defined by different algorithms and across various resolutions and parameters, further impeding a robust definition of these structures \cite{Dali.2017, Forcato.2017, Zufferey.2018}. While efforts have been made to functionally delineate between TADs at different scales \cite{Dixon.2016, Beagan.2020, Sikorska.2019, Szabo.2019}, most studies, especially those who rely solely on Hi-C data, do not make these distinctions.

Regardless of the challenge of defining TADs, it is clear that these 3D structures play an important role in genome organization and function \cite{Berlivet.2013, Dixon.2016, Beagan.2020, Andrey.2017, Franke.2018, Xie.2019, Dixon.2015, Delaneau.2019, Smith.2016}. Studies assessing the direct transcriptional effects of TADs have found mixed results, with some locus-specific work suggesting a strong impact of TAD disruption on gene expression \cite{Lupianez.2015, Hnisz.2016, Guo.2015, Laugsch.2019}, while other genome-wide results imply only mild effects on expression \cite{Ghavi-Helm.2019, Zuin.2014, Espinola.2020, Rao.2017}. Despite some uncertainty about the magnitude of regulatory changes induced by TAD disruptions, multiple independent lines of evidence suggest TADs are functionally relevant. Genes located within the same TAD can have strongly correlated expression patterns and are often coregulated during cell differentiation \cite{Nora.2012, Zhan.2017, Ramirez.2018}. TAD boundaries are strongly correlated with replication-timing domain boundaries \cite{Pope.2014}, and are enriched for insulator elements such as CCCTC-binding factor (CTCF) \cite{Dixon.2012, Rao.2014}. Disruptions in normative TAD structures have also been implicated in a number of human pathologies \cite{Lupianez.2016, Ibn-Salem.2014, Franke.2016}. While precise and robust TAD and boundaries definitions are still elusive, a general feature that is robust to the specific definition is that loci within a TAD make contact more frequently with other loci in the same TAD than with loci outside it. The common paradigm is that TADs represent insulated neighborhoods, constraining the possible set of interactions between cis regulatory elements (CREs) and target genes \cite{Andrey.2017, Franke.2018, Dixon.2015, Delaneau.2019, Smith.2016, Symmons.2014, Tanay.2013}. It is believed that TADs are critical features of the genome, serving to sustain specific sets of regulatory interactions while preventing ectopic interactions between regulatory elements and the wrong target genes \cite{Andrey.2017, Symmons.2014, Sexton.2015}.

Numerous other papers and reviews have delved into the history and functionality of TADs. Previous papers and reviews discuss the variance in algorithmic identification of TADs, the inconsistency of TAD calls at different resolutions, and the lack of robust approach to identify TAD boundaries based on Hi- C data \cite{Dixon.2016, Beagan.2020, Dali.2017, Forcato.2017, Zufferey.2018, Sikorska.2019, Andrey.2017, Franke.2018, Schoenfelder.2019, Rowley.2018, Fraser.2015, Ay.2015}. The notion that TADs are highly conserved across species and cell types is prevalent and often considered a foregone conclusion. Determining TAD variability across cell types is important for understanding the extent to which 3D genome structure affects differential gene regulation during development, enabling the regulatory and functional novelty observed in different cell lineages. In turn, assessing TAD conservation across evolution could help reveal the regulatory loci and mechanisms responsible for speciation and adaptation. We do not discuss further the issues related to similarities and differences in TADs across cell types and tissues, which have been previously discussed \cite{Andrey.2017, Sauerwald.2018, Sauerwald.2019, Schmitt.2016, Zheng.2019, Bonev.2016, Cubenas-potts.2015}. In this review, we wish to specifically focus on the notion that TADs and their boundaries are highly conserved across species.

Many studies are cited as reporting this conclusion, but it is difficult to trace the origin of this claim. If TADs and their boundaries are indeed highly conserved across spices, the origin of regulatory novelty must be elsewhere. However, if genome organization is not highly conserved, it is possible that changes in TADs and insulation boundaries may play an important role in underlying adaptation and speciation through changes in gene regulation. In other words, the answer to the question of TAD conservation has important implications for evolutionary research. We thus set out to thoroughly review the evidence for TAD conservation and we found that, in fact, only a few studies collected relevant data and provide direct evidence to support this notion.

\section{Conservation in Context}
In order to evaluate the evidence for conservation of TADs, it is important to consider what we actually mean when we refer to conservation of genetic and epigenetic features across species. At the level of a single feature{\textemdash}a single locus for example{\textemdash}conservation is easily defined when the state of the feature is identical across species. More generally, however, when studies refer to genome-wide properties, they typically do not use a specific standard for the definition of conservation. When studies refer to a general property (for example, chromatin accessibility) as conserved, it implies that this property evolves under natural selection to maintain similarity across species. However, very few studies formally test this hypothesis. In fact, for most functional genomic traits that are comparatively studied, we have not yet formulated as null model of `no selection.' 
Without a formal test, what do we typically mean when we conclude that a molecular trait is conserved? In most studies, conservation simply means `highly similar' across species. While this is typically not a formal process, the degree of similarity, or variance, is evaluated and benchmarked based on other relevant comparisons. For example, if the level of observed variation in a trait is similar within and between species, the trait is typically deemed highly similar across species and therefore conserved. If variation between species is consistently low regardless of the time to the most recent common ancestors of the species, the trait is likely to be conserved. If different molecular features show a range of inter-species variability, the features with the lowest variance across species are assumed to be conserved. All of these examples point to ad-hoc definition of conservation, but this does not mean that they are wrong.

Let us consider specific examples. Comparative studies have reported that genome-wide, the overlap of histone modification H3K4me3 locations in humans and chimpanzees is around 70\% \cite{Cain.2011}. Remarkably, the genome-wide overlap of H3K4me3 locations in humans and mouse is also around 70\% \cite{Woo.2012}. With these figures, not much could be said about the conservation of H3K4me3 locations in primates, but we can probably conclude with confidence that H3K4me3 locations are quite conserved between human and mouse. That said, the best genomic context for evaluating the degree of conservation in these comparisons may be other histone modifications, but even the minimal context provided here illustrates the importance of benchmarking similarity values in order to understand what they imply about conservation across species. 

To date, comparative studies of chromatin conformations and TADs did not put forward a formal null model with which to evaluate levels of conservation. Instead, statements regarding the conservation of TAD and boundaries were made by using the ad-hoc rationale we discussed above. With this in mind, we now turn to critically examine the existing evidence for evolutionary conservation of TADs.

\section{Indirect evidence for conservation}
The notion that TADs and are highly conserved appears to be supported by a number of studies. One class of such studies, however, does not perform direct comparative assessment of TADs and boundaries across species. Instead, the indirect inference of TAD conservation is based on comparative functional genomic data that are independently associated with TADs. The most common approach in this class of studies is to directly map TADs in one species, then infer the locations of TADs in other species based on genomic features that are associated with TADs, such as CTCF binding sites, high gene density, or regions of active transcription \cite{Dixon.2012, Nora.2012, Rudan.2015, Kentepozidou.2020, Dekker.2015}. To date, however, no single genomic feature can be used to effectively predict TAD locations and boundaries. Thus, the inference of TAD conservation based on other functional genomic features is indirect and might not be accurate. 

One of the most commonly cited studies supporting TAD conservation, Rudan et al. 2015 \cite{Rudan.2015}, used such an indirect inference approach. Rudan et al. 2015 \cite{Rudan.2015} collected comparative Hi-C data in liver cells from mouse, macaque, rabbit, and dog, but most of their comparative inference was based on the placement and orientation of CTCF binding sites. The authors conclusion that there is ``extensive genome-wide interspecies conservation of chromosome structure'' was based on comparisons of a broader set of contacts, not specifically of TADs. In fact, Rudan et al. did not report the total number of TADs identified in each species (only in mouse and dog), nor did they directly estimate the proportion of TADs that are found to be conserved across species. Instead, they used an indirect measure, estimating the interspecies correlations of inferred insulator activity \cite{Sofueva.2013} at different distances from orthologous genes. Rudan et al only reported species pairwise comparisons that involved the mouse data, resulting in Spearman correlations that ranged from 0.34 to 0.61. These correlations values may indicate some degree of conservation of 3D genome structure, but it is difficult to conclude from these analyses that TADs are indeed highly conserved across species. Moreover, Rudan et al. data collection was uneven across species, with ~275 million reads sequenced from mouse, ~150 million from rabbit, ~100 million from macaque, and ~550 million for dog. The large differences in read count result in a difference in the power to identify 3D genome structures across species and hence complicate the interpretation of the reported results. 

There are other widely cited studies that concluded that TADs are highly conserved based on indirect evidence. Harmston et al. 2017 \cite{Harmston.2017}, for instance, identified genomic regulatory blocks (GRBs, regions with a high density of conserved noncoding elements) in human, opossum, chicken, and spotted gar. They reported that GRBs are often quite conserved across species. Using previously collected Hi-C data from human and Drosophila, Harmston et al. have shown that GRBs often fall within TADs and/or have edges proximal to TAD boundaries in these two species. Based on these data, the authors concluded that TADs are generally conserved ancient features of the genome and that TAD boundaries are largely invariant between all  the species in their study. However, the data reported by Harmston et al. shows that only about a third of the TADs were associated with GRBs; thus, even if one accepts the indirect inference based on GRBs as correct, up two thirds of TADs may still not be conserved in these species, as no direct evidence for TAD conservation was presented in this study. Indeed, in their concluding statement, Harmston et al. are careful to note that only the subset of GRB-associated TADs appears to be ancient conserved structures. However, this paper is often cited as providing strong evidence for general TAD conservation across species.  

Similarly, Krefting et al. 2018 \cite{Krefting.2018} considered TADs that were previously directly identified in humans \cite{Dixon.2012, Rao.2014} along with genomic rearrangement breakpoints they identified in 13 species. Based on observed enrichment of these breakpoints at TAD boundaries and depletion within TAD bodies, the authors made relatively strong claims about TAD stability across evolutionary timescales. However, no direct comparison of TADs was made across species; the conclusions are essentially only based on the inter-species comparison of the rearrangement breakpoints. Given this, that the enrichments observed were only in comparison to TADs found in humans, and a fair amount of inconsistency in the degree of enrichment observed between the two different TAD sets used, we do not believe strong claims of conservation are warranted. Motivated by the findings of Harmston et al. 2017 \cite{Harmston.2017}, the same study also examined correlation of gene expression for orthologous genes in humans and mice within TADs associated with GRBs vs. orthologous genes in non-GRB associated TADs. While they found that gene expression within GRB-associated TADs is significantly slightly more correlated than the expression of non-GRB TAD genes, they also noted that more than 60\% of hESC TADs don't overlap GRBs, perhaps suggesting that only a small subset of TADs are actually conserved. The authors should be commended for making note of this, as well as including the caveat that the enrichment they observe may be due to chromatin accessibility differences between TAD boundaries and TAD bodies. Still, the majority of the paper makes claims of TAD conservation across evolutionary timescales, which simply are not supported due to inconsistent enrichments across different TAD sets and a lack of TAD inferences in any species aside from humans.

In another comparable approach, Lazar et al. 2018 \cite{Lazar.2018} chose to compare human and gibbon genomes in LCLs, given the large number of chromosomal rearrangements and high DNA sequence identity (96\%) between the two species. The authors cite previous studies to substantiate the claim that TADs are largely conserved across species \cite{Dixon.2012, Rudan.2015}, and ultimately conclude that most TADs have been maintained as `intact modules' during genomic divergence between humans and gibbons. Such claims may be overstated, given that this work did not undertake a direct comparison of TAD locations between the species, instead largely focusing on the overlap of multiple species\' TAD boundaries with 67 rearrangement breakpoints identified in the gibbon genome (in comparison to human). Though their results indicate a high overlap of TAD boundaries across multiple species with gibbon breakpoints above what would be expected at random, they do not, to us, indicate strong conservation of the TADs themselves across species. Our interpretation is further supported by the study\'s finding that only 19 of the 67 breakpoints (~28\%) overlapped TAD boundaries in all other species compared (human, rhesus, mouse, dog, and rabbit). Conversely, the authors found almost no evidence for new TADs being created between humans and gibbons based on rearrangement breakpoints, but we note again that this analysis is focused on the breakpoints rather than the TADs, and that the absence of evidence does not indicate evidence of absence.

Finally, we wish to highlight one exemplary study that also did not perform a direct assessment of TADs across species, but was measured in its conclusions and made a significant methodological contribution to comparative analysis of 3D genome topology. Yang et al. 2019 \cite{Yang.2019} represents one of the only papers with an explicit aim of comparing 3D genome organization across a number of primate species, and sequenced a similar number of Hi-C reads from LCLs in chimpanzees, bonobos, and gorillas, in addition to examining previously-published human Hi-C data in the same cell type \cite{Rao.2014}. The analysis framework they present for interspecies comparison\emdash phylo-HMRF (hidden markov random field)\emdash is sorely needed for interspecies comparative Hi-C.  The robustness of the method is also underscored by the proximity between its identified boundaries of Hi-C evolutionary pattern blocks and previously-inferred TAD boundaries \cite{Rao.2014}. We speculate that the authors focused on this overlap, rather than directly inferring TADs in the novel primate Hi-C datasets collected, because direct TAD inference and comparison would not be robust, just as general TAD inference is not robust. Despite similar sequencing depth across species, TAD detection would likely be confounded by differences in reference genome quality, a consideration the authors apparently took quite seriously, given that all Hi-C reads in the method are ultimately mapped back to the human reference genome. To our knowledge, this is one of the only papers presenting an analytical framework for interspecies Hi-C comparison, with the other being more focused on similarity of genomic contacts within orthologous TADs between species, rather than the locations of the TADs  \cite{Nuriddinov.2019}. As both of these methods are relatively recent,  they have not yet been more broadly applied to a wide range of Hi-C datasets across different species, but we look forward to such analyses being used in the future in our own work and that of others.

In summary, all of the studies we discussed in this section did not directly identify TADs in multiple species, could not perform a direct inter-species comparison of TADs, and thus these studies do not provide direct evidence for the notion that TADs are highly conserved. In order to truly understand the extent of conservation of these structures, we must infer and examine them across a number of species, and assess if a given TAD in one species has a corresponding counterpart in others. Otherwise, claims of conservation speak to conservation of features of TADs and 3D genome topology more broadly, rather than conservation of the individual structures themselves.

\section{Direct but anecdotal evidence for conservation}
The second class of studies that are widely cited as providing evidence for general TAD conservation provide only anecdotal evidence. These are studies that provide direct and strong evidence for conservation of TADs, but only in a small number of well-studied cases. It is thus difficult to generalize from these studies and conclude with confidence that TADs are generally highly conserved across species.

Woltering et al. 2014 \cite{Woltering.2014}, for example, found that Hox loci across zebrafish and mouse tend to have similar TAD structure, and Gomez-Marin et al. 2015 found comparable TAD structures across a number of species at the Six loci \cite{Gomez-marin.2015}. Both of these studies, as well as a number of others \cite{Smith.2016, Lupianez.2015, Galupa.2018, Galupa.2020}, focus on loci that are highly conserved and/or thought to be critical for normal organismal development. Though these findings underscore the functional importance of TADs, they do not provide evidence for broad and general TAD conservation. In particular, the focus on a subset of candidate loci that are more likely to contain conserved features makes it difficult to generalize these observations to a genome-wide scale. Thus, though these studies infer TAD conservation based on direct functional data, they do not provide strong support for the widely accepted notion that TADs are highly conserved. 

\section{Direct evidence for the conservation of TADs}
We now turn to the relatively small body of research that studied TAD conservation by directly identifying TADs and boundaries in multiple species. This direct approach seems the most obvious. In fact, it would be challenging to find another example in the genomics field where a widely accepted conclusion was mostly supported by indirect evidence and inference. In the case of comparative studies of TADs, only a few studies collected direct comparative data.

Dixon et al. 2012 \cite{Dixon.2012} collected Hi-C data and inferred TADs in human and mouse. This study was groundbreaking as it was one of the first to discover TADs and propose an algorithm to infer them from Hi-C contact maps (directionality index). This study is often cited as providing the first evidence that TADs are highly conserved between humans and mice. The authors collected 475 million sequencing reads from mouse Hi-C libraries but only 330 million reads from human. TAD boundaries were considered conserved if they had any overlap in the other species, with 76\% of mouse TAD boundaries found in humans but only 54\% of human boundaries found in mice. If one considers the entire dataset of TAD boundaries identified in both human and mouse (rather than the reported unilateral overlaps), ~31\% of boundaries are shared across the two species.

These results were and still are interpreted as evidence for strong TAD conservation. To provide some context, we considered other functional annotations in human and mouse. There is about 60-75\% overlap of loci marked by histone modification in humans and mice \cite{Woo.2012}, and between half to two-thirds of candidate regulatory regions are conserved in the two species \cite{Yue.2014}. Considering the observed proportion of overlapping TAD boundaries in human and mouse in this context, we believe that there is evidence for some level of conservation, but arguably, this cannot be considered strong evidence that TADs are generally highly conserved across species.

Another study that performed a direct comparative assessment of TADs is Rao et al. 2014 \cite{Rao.2014}. The authors collected ~6.5 billion Hi-C sequencing reads from human but only ~1.4 billion reads from mouse. The difference in read depth resulted in a striking difference in the power to infer TADs in the two species, with more than 9000 domains identified in human but only ~3000 domains found in mouse. The authors considered entire domains conserved if the center of a domain in one species was within 50 kb of an annotated domain in the other species (or within half the domain size, for domains smaller than 100 kb). Ultimately, Rao et al. 2014 reported that 45\% of mouse domains (where they had considerably less power to identify TADs) were also present in human. Again, there may be some evidence for conservation, but it is difficult to conclude based on these data that TADs are highly conserved. 

As far as we know, Dixon et al. 2012 and Rao et al. 2014 are two of the only studies that concluded that TADs are highly conserved based on a direct analysis of TADs and boundaries in more than one species. Both works used human and mouse, and utilized an unbalanced sequencing study design across species, which makes the interpretation of the results somewhat challenging. Regardless, even if we accept the observation of Dixon et al. 2012 and Rao et al. 2014 at face value, the reported overlap of TADs and boundaries in human and mouse arguably does not indicate that these features are highly conserved.

\section{On the other hand...}
There are a few studies that suggest that TADs may not be particularly conserved across species. Berthelot et al. 2015 \cite{Berthelot.2015} considered the order of orthologous genes to identify genomic rearrangement breakpoints in the genomes of human, mouse, dog, cow, horse, and a genomic reconstruction of the Boreoeutherian last common ancestor. In an attempt to understand the non-random genomic distribution of these breakpoints, the authors considered the overlap of rearrangement breakpoints with TADs that were previously identified in human \cite{Dixon.2012}. Because the basal set used for comparisons was breakpoints rather than TAD boundaries, this is another example of a study that relied on indirect inference. In contrast to the results described from Lazar et al. \cite{Lazar.2018}, the authors did not find evidence for strong overlap of TAD boundaries and breakpoints, reporting that only 8\% of the identified breakpoints overlap with TAD boundaries. This would suggest that TADs do not generally contribute to the locations of genomic rearrangements.

Berthelot et al. 2015 \cite{Berthelot.2015} is also notable for its interpretation of the results of Dixon et al. 2012 \cite{Dixon.2012}. While the vast majority of papers cite Dixon et al.\'s study as providing strong evidence that TADs are highly conserved, Berthelot et al. cite the same study to provide evidence for some TAD divergence between humans and mice. That the results of Dixon et al. 2012 can be interpreted by different groups both as supporting conservation of TADs or lack thereof highlights our notion that the foundation for the claim that TADs are highly conserved is not strong.

The notion that TADs may not be particularly conserved is also supported by our own study, in which we directly inferred TADs in humans and chimpanzees \cite{Eres.2019}. Our initial analysis found only ~43\% of TADs conserved between these species, but across many different parameters (e.g. resolution, window size, genome assembly), and different downstream analysis decisions, we found that no more than 78\% of domains and 83\% of TAD boundaries were shared between humans and chimpanzees{\textemdash}a much lower percentage than what has been seen across these species for a number of other functional regulatory phenotypes.

The notion that TADs may not be particularly conserved is also supported by recent results from Hi-C data across three distantly related Drosophila species. Renschler et al. 2019 \cite{Renschler.2019} inferred TADs and genomic rearrangements across three Drosophila species, and found significant overlap above what would be expected by chance alone. However, the percentages of TAD boundaries overlapping a rearrangement breakpoint were relatively low (ranging from 13-21\% of boundaries depending on the comparison). The proportion of overlapping TADs was even lower, at 10\%.
 
Other findings, particularly in plants, also suggest that TAD positions may not be conserved across species. Dong et al. 2017 \cite{Dong.2017}, for instance, collected Hi-C data from maize, tomato, sorghum, foxtail millet, and rice, and found relatively little conservation of TADs across these species. Although plants lack a homolog for CTCF, a transcription factor strongly implicated in the maintenance of TAD boundaries \cite{Guo.2015, Rudan.2015, Kentepozidou.2020, Gomez-marin.2015}, the authors observed TAD-like domains in contact maps across all species, and found that they share many epigenetic features with TADs inferred in mammals. Xie et al. 2019 \cite{Xie.2019} used a similar method to assess TAD conservation in two different mustard plants, Brassica rapa and Brassica oleracea, and reported that about 25\% of all TADs are found in both species.

It should be noted that the existence of TADs in plants, worms, yeast, and other non-mammalian species is a matter of active debate \cite{Bonev.2016}. While chromatin conformation capture experiments have revealed self-interacting TAD-like structures in many of these species, their characteristics and mechanisms of formation often differ substantially from those of mammalian TADs \cite{Szabo.2019, Acemel.2017}. In many cases, these species lack homologs for insulator proteins thought to be essential to the formation of mammalian TADs (e.g. CTCF) \cite{Szabo.2019}. More samples and more deeply sequenced Hi-C libraries from these species, as well as a deeper understanding of possible mechanisms of TAD-like feature formation, will be necessary to thoroughly assess conservation of TAD structures across all of evolution.

\section{Concluding remarks and future perspectives}
It is important to note that we are not taking a strong position `for' or `against' the notion of TAD conservation. Based on the available evidence, we conclude that there is currently no satisfying answer to the question of just how conserved TADs are across evolution. While the results from certain studies suggest some degree of conservation, others often lead to much lower estimates, and flawed study designs and variable analytical choices further obscure the issue. Although many studies state that TADs are conserved across species, there are only sparse data supporting or refuting this claim. In our mind, there is no strong basis for the common and often unchallenged notion that TADs are highly conserved. 

One of the largest factors affecting our ability to assess evolutionary TAD conservation is the lack of any `gold standard,' either for inferring TADs or for comparing them across species. As others have noted, TADs are variously and poorly defined, and it seems likely that stable TADs observed in Hi-C data represent statistical features that emerge from averaging more dynamic interactions across millions of cells \cite{Wit.2019}. The few studies that did directly compare TADs across species made somewhat arbitrary choices about how to call these features conserved.

We struggled with this and many other aforementioned issues in our own work examining 3D genome structure across humans and chimpanzees \cite{Eres.2019}. Despite our own results suggesting a fair degree of TAD divergence between the species, we were unable to find many clear visual examples where divergent TAD inferences were obvious, based on the contact map. This once again emphasizes the need for specific, robust analytical methods to compare 3D genome topology and infer TADs across species . Unfortunately, evolutionary TAD conservation may remain an open and evolving question until we arrive at a more precise definition of TADs and converge on a set of truly robust methods for TAD inference and comparison.

To be clear, we do not disagree with the notion that a subset of TADs, particularly those involved in the regulation of key developmental loci or found near genomic rearrangement breakpoints, are likely to be highly conserved across species. We simply disagree with the conclusion often made, based on TAD subsets and existing interspecies comparative data, that TADs are highly conserved across species. Certainly, the existing evidence suggests that TADs as functional units of 3D genome organization exist and have similar epigenetic features across many different species. In mammals, a copious amount of chromatin contact data suggests some degree of conservation of TAD structure. However, the existing direct comparative data and analyses do not, in our opinion, provide enough evidence to claim strong conservation of TAD positioning across evolution.

Future studies hoping to assess 3D genome conservation across species should attempt to use a wide variety of TAD algorithms and parameters, as well as new interspecies Hi-C analytical methods to assess 3D genome conservation \cite{Yang.2019, Nuriddinov.2019}. Research addressing this question should also take great care to sequence a similar number of reads across species, and check the robustness of their results across different analytical decisions for calling TADs and their boundaries conserved. TADs represent one intriguing feature of 3D genome architecture, and evolutionary conservation of other features (e.g. regulatory loops) is even less clear. In order to understand regulatory dynamics overall, we must refine our understanding of TADs, and agree on how to infer and compare them across species and cell types.

\section{Acknowledgments}
We apologize to the authors of relevant studies whose work was not addressed due to space limitations. We thank Natalia Gonzales, Jasmin Zohren, Sergey Kolchenko, Daniel Ibrahim, and Carlos Bustamante, for useful discussions, comments, and/or edits to the manuscript. YG is supported by NIH grant R35GM131726.


