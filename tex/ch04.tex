\chapter{Conclusion}\label{conclusion}

\section{Evolutionary and gene regulatory implications of this work}
The field of human genetics has already accomplished much in connecting genetic variation to complex trait and disease variation between individuals.Outstanding challenges remain in characterizing the mechanisms of action for all these variants, and understanding the relative contributions of these different mechanisms to speciation, adaptation, and inter-individual trait variation. Although our understanding is rapidly expanding, we are still far from obtaining a comprehensive picture of gene regulation. The findings detailed herein corroborate the idea that assessing 3D genome structure is a crucial next piece of the puzzle. In Chapter 2, collaborators and I measured 3D genome structure and gene expression across human and chimpanzee induced pluripotent stem cells (iPSCs), revealing that differences in 3D genome structure may contribute to differential gene expression across these species. We found that, at the lowest scale (i.e. individual gene regulatory DNA loops), human and chimpanzee chromatin contacts are fairly conserved. This would imply that, at least in iPSCs, individual gene-cis-regulatory element (CRE) interactions do not vary significantly between humans and chimpanzees. However, this does not necessarily mean that the origins of regulatory novelty must lie elsewhere. Chromatin state in iPSCs is generally more open and "permissive" than in differentiated cell types \cite{Spivakov and Fisher 2007 (Epigenetic signatures of stem-cell identity)}, and thus we may expect lower divergence in gene-CRE loops across species in iPSCs than in other cell types. Interestingly, we also observed evidence for large sets of species-biased differences in loop strength on individual chromosomes that have experienced large-scale rearrangements between humans and chimpanzees. This suggests that such rearrangements may help drive interspecies regulatory novelty in 3D chromatin interactions, although more functional follow-up will be required to confirm this notion, given conflicting results from some other comparative studies \cite{Lazar et al. 2018 (Epigenetic maintenance of topological domains in the highly rearranged gibbon genome), Krefting et al. 2018 (Evolutionary stability of topologically associating domains is associated with conserved gene regulation)}.

Regardless of the precise level of conservation/divergence in DNA looping, we found ample evidence for pairs of loci exhibiting differential contact (DC) across species. When we overlapped these data with RNA-seq data assessing gene expression levels, we observed strong correlations between interspecies contact and expression differences for differentially expressed (DE) genes overlapping our Hi-C loci. The fact that we did not observe similarly strong correlations for non-DE genes implies that variation in chromatin contacts plays a role in DE. Further corroborating this notion, we observed a significant enrichment for DE amongst genes we classified as DC across species--and vice versa. As previously stated, the observational nature of the study meant we could not directly infer a causal relationship between DC and DE. However, our mediation analysis found that up to 8\% of DE genes may have a significant portion of their expression variation explained by variation in chromatin contacts. Placed in the context of other studies that observed perturbations in chromatin contact affecting gene expression \cite{Lupianez et al. 2015 (Disruptions of topological chromatin domains cause...), Siersb�k et al. 2017 (Dynamic rewiring of promoter-anchored chromatin loops during adipocyte differentiation)}, our findings suggest that species-specific differences in 3D genomic contacts are indeed a driver of species-specific expression. This conclusion is also supported by our observation that, compared to contacts not involving a promoter, promoter-associated contacts are enriched for more active chromHMM state assignments. The intuitive conclusion from this result is that loci making contact with a promoter are likely involved in active regulation of the corresponding gene. Similarly, we observed that joint DE/DC loci identified in our study are enriched for a wide variety of functional epigenetic marks as compared to non-DE/DC loci. Under the common paradigm that most chromatin is not accessible and thus not active in any given cell type \cite{Thurman et al. 2012 (The accessible chromatin landscape of the human genome)}, these functional annotation enrichments suggest that the identified joint DE/DC loci represent functionally relevant stretches of DNA between the species. Based on all these results, it may be tempting to speculate that 3D genome conformation is one of the most basal elements laying the groundwork for a broad cascade of events dictating gene regulation. Unfortunately, this conclusion seems somewhat premature absent a bevy of mechanistic perturbation studies, and given more recent conflicting results about the order and nature of events with respect to genome conformation and observed differences in gene expression \cite{Jiang et al. 2020 (Genome-wide analyses of chromatin interactions after the loss of Pol I, Pol II, and Pol III), Ghavi-Helm et al. 2019 (Highly rearranged chromosomes reveal uncoupling between genome topology and gene expression), Espinola et al. 2020 (biorxiv, Cis-regulatory chromatin loops arise before TADs and gene activation, and are independent of cell fate during development), Ing-Simmons 2020 (biorxiv, Independence of 3D chromatin conformation and gene regulation during Drosophila dorsoventral patterning), Alexander et al. 2019 (Live-cell imaging reveals enhancer-dependent Sox2 transcription in the absence of enhancer proximity), Benabdallah et al. 2019 (Decreased Enhancer-promoter proximity accompanying enhancer activation)}. Although cause and consequence are still difficult to disentangle in this framework, there is no doubt that 3D genome conformation is an important facet affecting the evolution of gene regulation. 

Beyond exploring regulatory loop conservation and its effects on expression, the data collected in Chapter 2 also enabled us to examine higher-order chromatin structure, such as topologically associating domains (TADs), across the species. We found relatively weak conservation of TAD structures as compared to regulatory loops and other epigenetic phenotypes previously compared between humans and chimpanzees. While this might point to TAD variation as a significant source of regulatory novelty, we were unable to find concrete examples of interspecies TAD differences affecting differential expression. This does not, however, preclude a significant role for TAD variation in speciation and interspecies expression divergence; as I discuss further below, this lack of signal could be due to TADs being poorly defined and difficult to robustly infer \cite{Dali and Blanchette 2017}. Regardless, the observed low interspecies TAD conservation was surprising, given the prevailing notion in the field that TADs are highly conserved across species \cite{Dixon et al. 2012, Rao et al. 2014}. In large part, this incongruence motivated our critical assessment of the evidence for evolutionary TAD conservation, detailed in Chapter 3. A thorough review of the available data suggest that, while there is certainly some evidence for TAD conservation across mammalian species, it is not compelling enough to claim TADs are highly conserved. The validity of this notion is important to consider because, if true, it implies that TAD variation does not play a significant role in speciation. As addressed further in the final section of this chapter, analytical and definitional issues stymie a robust assessment of interspecies TAD variation and preclude a thorough understanding of TADs' impact on gene expression. This much is evident from the fact that, although the TAD inference algorithms we employed found numerous differences between the species, visualizations of the corresponding contact maps did not appear significantly different between humans and chimpanzees. Thus, the results of our own TAD comparisons do not necessarily conflict with previous findings. It is possible that differences in TADs play a significant role in differences observed between primate species, but it is difficult to support or refute this notion with any confidence, given the current state of the field. Despite this, my thesis work has broadly confirmed the idea that 3D genome organization is an integral feature affecting the evolution of gene regulation. At the same time, it is important to understand the limitations of this work, and consequently, avenues for future research.

\section{Limitations and next steps}
There are a number of limitations to this research program that should be considered to inform avenues for future research. For one, the true extent of interspecies divergence in gene-CRE interactions may be higher than our estimate, given the underpowered nature of the Hi-C assay \cite{Belton et al. 2012 (Hi-C: A comprehensive technique to capture the conformation of genomes)} and our limited number of individuals from each species. Still, our analysis was carried out in a robust quantitative fashion that is likely to give more accurate estimates of inter-species conservation than simplistic approaches other studies have used (i.e. assessing conservation via a venn diagram of overlap in significant chromatin contacts per-species) \cite{Dixon 2012, Rao 2014}. Directly testing each contact identified as significant in any individual for inter-species differences allowed us to largely sidestep the issue of incomplete power, avoiding an overinflated estimate of divergence. While numerous methods have been proposed to quantitatively compare regulatory loop strength across different biological conditions \cite{Lun and Smyth 2015 (DiffHiC), Paulsen et al. 2014 (HiBrowse: multi-purpose statistical analysis of genome-wide chromatin 3D organization), Djekidel et al. 2018 (FIND: difFerential chromatin INteractions Detection using a spatial poisson process), Fernandez et al. 2020 (3DeFDR: statistical methods for identifying cell type-specific looping interactions in 5C and Hi-C data), Rao et al. 2014}, sparse novel techniques have only very recently emerged for running similar comparisons across species \cite{Yang et al. 2019 (Comparing 3D genome organization in multiple species using phylo-hmrf), Nuriddinov and Fishman 2019 (C-intersecture--a computational tool for interspecies comparison of genome architecture)}. When we began analyzing the data collected in Chapter 2, these techniques had not yet been published, but applying them to these and similar data in the future would be of great interest for robustly assessing primate divergence in regulatory chromatin looping. In a similar vein, the sequencing depth in our own study allowed for assessment of chromatin contacts at a 10 kb resolution, but a comprehensive picture of inter-primate differences in chromatin loops will require deeper sequencing to enable sub-kilobase resolution and analysis of finer-scale loops. Lastly and as noted above, our comparisons were only performed in iPSCs, which tend to have more permissive regulatory landscapes than differentiated cell types \cite{Spivakov and Fisher 2007 (Epigenetic signatures of stem-cell identity)}. Comparison of 3D chromatin structure across species in other cell types would thus be highly desirable, and may reveal greater divergence in gene-CRE loops than that observed in our own work. Ideally, this would be performed on isogenic samples to reduce the confounding effects of genetic variation, which could be accomplished by differentiating the same iPSC lines into a variety of terminal cell types.

Another important limitation to consider is that, when examining functional enrichments, we only overlapped our DE/DC loci with publicly-available epigenetic data from humans. The precise functional significance and evolutionary impact of these loci could be more thoroughly assessed and polarized in the future by adding chimpanzee epigenetic mark data. Such an analysis would be particularly interesting for assessing which DC loci have undergone differential CRE evolution between species, and thus are more likely to have species-specific effects on gene regulation. The mediation analysis we utilized to assess the impact of DC on DE could be expanded to include epigenetic mark data across species, improving power to predict gene expression differences \cite{Karlic et al. 2010 (Histone modification levels are predictive for gene expression)}. More broadly, our analyses integrating Hi-C data and RNA-seq data could be improved in a number of ways that may find more expression variation explained by chromatin contact variation. As is also the case for assessment of conservation, deeper sequencing could provide better resolution of individual gene-CRE interactions, making it easier to tease out the effects of chromatin contact on expression. At the 10 kb resolution we used, there were instances of multiple genes being assigned to the same Hi-C bin, likely obscuring interesting signals that could be observed in finer-scale data. Similarly, greater signals of association between chromatin contact and expression might have been observed if the RNA-seq and Hi-C data were collected concomitantly. Although our RNA-seq data came from the same cell lines, they were collected previously by different researchers culturing the cells in slightly different conditions. Concomitant collection would be more likely to maintain (and thus detect) weaker links between chromatin conformation and gene expression that may have been concealed in our own data. In some sense, our observational collection of these data across species represents an experimental paradigm for "natural perturbation" of chromatin structure and gene expression. At the same time, a thorough understanding of DC affecting DE would require more precise targeted perturbations, altering regulatory loop strength in one species and expecting to see corresponding expression "rescue" to comparable levels observed in the other species.

Our inferences regarding conservation of TAD structure could also be improved upon with functional and perturbational follow-up studies. As mentioned in the previous section, our algorithmic inference of TAD divergence often appeared fairly conserved upon visual inspection. This is probably largely due to issues with TAD identification (discussed further below), but could be mitigated with further functional characterizations. In particular, divergence in TAD structure could be more confidently characterized by also collecting ChIP-seq for CCCTC-binding factor (CTCF), a protein centrally involved in anchoring chromatin loops and demarcating TAD boundaries \cite{Rao et al. 2014, Zuin et al. 2014 (Cohesin and CTCF differentially affect chromatin architecture and gene expression in human cells), Phillips-Cremins et al. 2013 (Architectural protein subclasses shape 3D organization of genomes during lineage commitment), Dixon et al. 2012, Nora et al. 2012 (Spatial partitioning of the regulatory landscape of the X-inactivation centre)}. Overlaying these data with Hi-C data could help differentiate between instances of TAD divergence that are reflective of true biology (i.e. their boundaries show differences in CTCF binding across species), and divergence that is driven by technical issues (i.e. minimal difference in CTCF binding, but the inference algorithm fails to detect a TAD in one species where it nonetheless appears present). Situations falling into the former category could be further validated and characterized via CRISPR-based perturbations to the differential CTCF binding site in both species. Specifically, one could abrogate the binding site in the species where CTCF is strongly bound, verify reduced binding via ChIP-seq, and observe subsequent effects on TAD structure and corresponding gene expression. Carrying out the reciprocal experiment (i.e. creating a binding site in the species where CTCF is not bound) would also be useful. In both instances, these perturbations should drive TAD structure and local gene expression to appear more similar to the species where CTCF binding was not altered. If this expectation ends up being incorrect, that is in and of itself an extremely interesting result, and could spur further research into the mechanisms driving TAD formation and divergence across species. Furthermore, integrating the other data types collected with CTCF occupancy differences across species would add another intriguing layer of regulation that could help explain individual instances of DC and DE, and their effect on one another. Lastly, although it has been examined in some previous studies within a single species \cite{Sauerwald et al. 2019 (Analysis of the structural variability of topologically associated domains as revealed by Hi-C), Sauerwald and Kingsford 2018 (Quantifying the similarity of topological domains across normal and cancer human cell types), Schmitt et al. 2016 (A compendium of chromatin contact maps reveals spatially active regions in the human genome), Dixon et al. 2015 (Chromatin architecture reorganization during stem cell differentiation)}, estimating TAD sharing between tissues and cell types across primate species would be of great interest for elucidating the role these structures may play in differentiation and development, and how this may differ across evolution. Unfortunately, as I discuss in Chapter 3 and elaborate upon below, robust inferences regarding TAD conservation, function, and tissue-specificity are still severely hampered by imprecise and variable TAD definitions.

\section{The state of the 3D genome field, challenges, and future perspectives}
3D genomics is an exciting and relatively young field of epigenetic research. While the novelty of the field allows for high-impact discoveries and inferences, it also presents unique challenges. Hi-C is regarded as a revolutionary technology enabling genome-wide assessment of 3D genomic contacts, but it is also known to have a poor signal:noise ratio \cite{Lajoie et al. 2016 (The Hitchhiker's Guide to Hi-C Analysis: Practical guidelines), Yardimci et al. 2019 (Measuring the reproducibility and quality of Hi-C data)}. This has improved somewhat as researchers have transitioned away from dilution Hi-C to in-nucleus Hi-C, reducing the number of spurious interchromosomal trans reads \cite{Nagano et al. 2015 (Comparison of Hi-C results using in-solution versus in-nucleus ligation)}, but still remains a problem. Several factors contribute to this issue. The method creates chimeric molecules by ligating proximal restriction fragments together, and the number of possible pairwise fragment interactions is very high, regardless of the restriction enzyme used. This means that, in order to achieve statistical power to detect significant contacts, reads from Hi-C data must typically be binned into fixed-size intervals tiling the genome \cite{Pal et al. 2018 (Hi-C analysis: from data generation to integration)}. Even when sequencing depth is great enough to achieve individual restriction fragment resolution, these fragments are often not the same size, resulting in differences in power to infer contact at different loci \cite{Yaffe and Tanay 2011 (Probabilistic modeling of Hi-C contact maps eliminates systematic biases to characterize global chromosomal architecture)}. Differences in restriction fragment length, chromatin accessibility, and GC content also affect the efficiency of ligation, restriction enzyme cutting, and sequence amplification, respectively, exacerbating power differences between restriction fragments \cite{Yaffe and Tanay 2011 (Probabilistic modeling of Hi-C contact maps eliminates systematic biases to characterize global chromosomal architecture)}. In addition, proximity ligation can introduce spurious ligation products (e.g. self-circularized ligations) that add more noise, as they do not actually represent chimeric molecules connecting two linearly distant loci in physical proximity \cite{Belaghzal et al. 2017 (Hi-C 2.0: An optimized Hi-C procedure for high-resolution genome-wide mapping of chromosome conformation)}. That Hi-C data have many sources of systematic bias is evident from the plethora of studies proposing models to analyze Hi-C and address these biases, both explicitly (normalizing for specific sources of bias) and implicitly (normalizing the data to achieve equal visibility across loci) \cite{Yaffe and Tanay 2011 (Probabilistic modeling of Hi-C contact maps eliminates systematic biases to characterize global chromosomal architecture), Hu et al. 2012 (HiCNorm: removing biases in hi-C data via Poisson regression), Imakaev et al. 2012 (Iterative correction of hi-C data reveals hallmarks of chromosome organization), Rao et al. 2014 (A 3D map of the human genome at kilobase resolution reveals principles of chromatin looping), Cournac et al. 2012 (Normalization of a chromosomal contact map), Knight and Ruiz 2013 (A fast algorithm for matrix balancing), Lin et al. 2012 (Global changes in nuclear positioning of genes and intra- and inter-domain genomic interactions that orchestrate B cell fate), Stansfield et al. 2018 (HiCcompare: an R-package for joint normalization and comparison of Hi-C datasets), Wingett et al. 2015 (HiCUP: pipeline for mapping and processing Hi-C data), Servant et al. 2015 (HiC-Pro: an optimized and flexible pipeline for Hi-C data processing), Sauria et al. 2015 (HiFive: a tool suite for easy and efficient HiC and 5C data analysis)}. A very recent paper proposed a different method to assess chromatin organization that seems less noisy than Hi-C data, but its full utility is difficult to ascertain before it sees more widespread use \cite{You et al. 2020 (Direct DNA crosslinking with CAP-C uncovers transcription-dependent chromatin organization at high resolution)}. Future research should endeavor to use this and other methods that may arise to comprehensively assay 3D genome structure with fewer sources of bias, but, for the time being, Hi-C remains the dominant technique. Newer techniques could also be useful for characterizing cell-to-cell variability in 3D genome structure; single-cell methods exist for Hi-C, but they are plagued by issues with data sparsity, genome coverage, and incomplete power, even moreso than bulk Hi-C \cite{Ramani et al. 2017 (Massively multiplex single-cell Hi-C), Ramani et al. 2019 (Sci-Hi-C: A single-cell Hi-C method for mapping 3D genome organization in large number of single cells), Nagano et al. 2013 (Single-cell Hi-C reveals cell-to-cell variability in chromosome structure), Nagano et al. 2015 (Single-cell Hi-C for genome-wide detection of chromatin interactions that occur simultaneously in a single cell), Zhou et al. 2019 (Robust single-cell Hi-C clustering by convolution- and random-walk-based imputation)}.

In much the same way that there is no single agreed upon method to address Hi-C biases and normalize the data, the field also lacks a "gold standard" method for assessing significant chromatin contacts. A wide variety of statistical paradigms have been proposed, but no single method stands out. Studies comparing significance calling algorithms typically recommend researchers choose an algorithm that will work well for their downstream analyses, given differences in the quantity and characteristics of significant loops identified by each option \cite{Forcato et al. 2017 (Comparison of computational methods for Hi-C data analysis), Ay and Noble 2015 (Analysis methods for studying the 3D architecture of the genome), Lyu et al. 2019 (Comparison of normalization methods for Hi-C data)}. There is thus a pressing need to converge upon a "gold standard" method for identification of significant interactions. Sadly, such a convergence does not appear likely any time soon, since the field cannot even agree upon a standard format for storing Hi-C data, let alone analyzing it \cite{Pal et al. 2019 (Hi-C analysis: from data generation to integration), Marti-Renome et al. 2018 (Challenges and guidelines toward 4D nucleome data and model standards)}. In the meantime, much as was done in Chapter 2, studies seeking to examine significant chromatin contacts should utilize a number of different algorithms, in order to ensure their resulting inferences are robust. It is important to note that the field of 3D genome research has at least arrived at a (relative) consensus definition of what constitutes a significant interaction: a pair of loci with Hi-C reads connecting them more often than would be expected by chance, given their linear genomic distance. The lack of agreement seems centered primarily on how to statistically assess the significance of these interactions, and, relatedly, what is an appropriate null model for "no significant contact." Similarly, although finer demarcations of A/B compartments have been observed with higher-resolution Hi-C data \cite{Rao et al. 2014 (A three-dimensional map of the human genome at kilobase resolution reveals principles of chromatin looping)}, the field appears to largely agree upon the broad nature of A/B compartments (active/inactive chromatin), and the class of methods used to identify them (i.e. principal components analysis and clustering) \cite{Miura et al. 2018 (Practical Analysis of Hi-C data: Generating A/B compartment profiles)}. Thus, although sometimes disparate methods exist to quantify the 3D genome at these two scales (chromatin loops and A/B compartments), there is at least some consensus about the foundational definitions of these features. As alluded to in Chapter 3, no such consensus exists with respect to TADs.

Indeed, many of the aforementioned issues with TADs stem from the lack of a clear definition. When TADs were first discovered, they were defined in an analytical (rather than biological) fashion: as large squares of enhanced contact frequency arising off the diagonal in Hi-C maps \cite{Dixon et al. 2012, Nora et al. 2012 (Spatial partitioning...), Hou et al. 2012 (Gene density, transcription and insulators...), Sexton et al. 2012 (Three-dimensional folding and functional organization...)}. At the relatively low resolution of these original studies (40 kb), TADs emerged as megabase-scale, non-overlapping, highly self-interacting regions of the genome. Importantly, loci within a TAD not only make contact with other loci in the same TAD more often, but appear somewhat insulated from making contacts with loci outside of the TAD. TADs were systematically inferred with a hidden Markov model based on a "directionality index" that quantified the degree of upstream or downstream contact bias at each Hi-C bin, under the intuition that TAD boundaries should display sharp transitions in this bias state \cite{Dixon et al. 2012}. While subsequent studies achieved improved Hi-C resolution with greater sequencing depth and identified nested and overlapping TADs at much smaller scales, they were still defined based on technical features of the data \cite{Rao et al. 2014}. Despite the discovery that TADs exist at different scales, many novel TAD inference algorithms proposed in the years since have not made these distinctions, and are generally billed as TAD predictors, irrespective of scale \cite{Dali and Blanchette 2017}. Some TAD inference algorithms have been built for analysis of specific TAD sizes and hierarchies \cite{Rao et al. 2014, Weinreb et al. 2016 (Identification of hierarchical chromatin domains)}, but most have not. This is problematic because different resolutions and algorithmic parameters are necessary for robust detection of different hierarchies of TADs: a low-resolution Hi-C experiment might easily be able to detect megabase-scale TADs (sometimes termed "meta-TADs" \cite{Fraser et al. 2015 (Hierarchical folding and reorganization of chromosomes are linked to transcriptional changes in cellular differentiation)}), but will not have sufficient power to detect smaller TADs at the scale of several hundred kilobases (sometimes termed "sub-TADs" \cite{Phillips-Cremins et al. 2014 (Architectural protein subclasses shape 3-D organization of genomes during lineage commitment)}). Overall, identification of TADs remains an outstanding issue, as highlighted by a number of studies that have found low concordance between different TAD algorithms \cite{Dali and Blanchette 2017, Zufferey et al. 2018 (Comparison of computational methods for the identification of TADs), Forcato et al. 2017 (Comparison of computational methods for Hi-C data analysis)}. It may be tempting to blame these issues on the algorithms themselves, but the core of the problem truly resides in the definition of a TAD.

In the years since their discovery, TADs have been functionally characterized in many ways. TAD boundaries are enriched for CTCF and cohesin binding sites \cite{Dixon et al. 2012, Rao et al. 2014, Van Bortle et al. 2014 (Insulator function and topological domain border strength scale with architectural protein occupancy)}, active histone marks, and transcription start sites (TSS) of housekeeping genes \cite{Szabo et al. 2019 (Principles of genome folding into topologically associating domains), Hou et al. 2012 (Gene density, transcription, and insulators contribute to the partition of the Drosophila genome into physical domains), Ramirez et al. 2018 (High-resolution TADs reveal DNA sequences underlying genome organization in flies)}. The functional significance of TADs is highlighted by these findings, as well as the observations that genes within the same TAD exhibit strongly correlated expression patterns \cite{Nora et al. 2012 (Spatial partitioning...), Ramirez et al. 2018 (High-resolution TADs...), Symmons et al. 2014 (Functional and topological characteristics of mammalian regulatory domains)}, and that enhancer-promoter contacts largely occur within the same TAD \cite{Bonev et al. 2017 (Multiscale 3D genome rewiring during mouse neural development), Symmons et al. 2016 (The Shh topological domain facilitates the action of remote enhancers by reducing the effects of genomic distances), Smith et al. 2016 (Invariant TAD boundaries constrain cell-type-specific looping interactions between...), Delaneau et al. 2019 (Chromatin three-dimensional interactions mediate genetic effects on gene expression)}. There has also been great interest in elucidating the mechanisms behind TAD formation. Outstanding questions remain, but accumulating evidence suggests these structures are largely formed via two mechanisms: loop extrusion, and compartmentalization \cite{Nuebler et al. 2018 (Chromatin organization by an interplay of loop extrusion and compartmental segregation), Fudenberg et al. 2016 (Formation of chromosomal domains by loop extrusion), Sanborn et al. 2015 (Chromatin extrusion explains key features of loop and domain formation in wild-type and...), Rowley et al. 2017 (Evolutionarily conserved principles predict 3D chromatin organization), Eagen et al. 2017 (Polycomb-mediated chromatin loops revealed by a subkilobase-resolution chromatin interaction map)}. Despite all these functional and mechanistic characterizations, the definition of a TAD has not changed much. Some recent reviews have proposed refining TAD definitions based on scale, overlap with other 3D chromatin features, and putative mechanisms of formation \cite{Dixon et al. 2017 (Chromatin domains: the unit of chromosome organization), Beagan and Phillips-Cremins 2020 (On the existence and functionality of topologically associating domains)}, but the field has not yet widely adopted these distinctions. Such delineations will be crucial to furthering our understanding of 3D genome structure and its regulatory effects moving forward, particularly in light of recent observations in single cells that suggest TADs are much more dynamic than originally thought \cite{Gizzi et al. 2020 (TADs or no TADs: Lessons from single-cell imaging of chromosome architecture), Finn et al. 2019 (Extensive heterogeneity and intrinsic variation in spatial genome organization)}. These and other findings imply that TADs identified from bulk Hi-C data may be statistical artifacts that emerge from averaging chromatin conformation in millions of cells \cite{De Wit 2019 (TADs as the Caller Calls Them)}, further underscoring the need for updated TAD definitions that reflect differences in specific factors and/or mechanisms involved in their formation. In the future, I suggest researchers combine mechanistic, functional, and single-cell techniques to thoroughly characterize different classes of TADs, ideally giving them different names. This would be a tremendous boon in helping define a "ground truth" for TADs, against which to test the output of various inference algorithms. It may be infeasible to functionally characterize TADs genome-wide, but perhaps higher-resolution research and more widespread use of the distinctions already proposed could reveal facets of the data that are sufficient for distinguishing different classes of TADs from one another \textit{in silico}. Until then, future studies assessing TAD structure should consider utilizing a wide array of different algorithms, in order to establish a confident set of TADs. Simultaneously, I hope more work emerges using CRISPR and other techniques to carry out perturbational follow-up experiments assessing the impact of specific TADs on gene regulation across a variety of cell types, conditions, and species. When it comes to TADs, a vast frontier of exciting discoveries clearly remains.

3D genomics has done much to expand our understanding of the evolutionary and developmental aspects of gene regulation, but more precise definitions and robust methods will be necessary to ensure continued impact moving forward. As the cost of sequencing continues to decrease, more studies will hopefully be able to assess chromatin conformation in larger panels with more individuals. One recent seminal work using 20 individuals found quantitative trait loci (QTL) that affect several facets of higher-order 3D genome structure, dependent upon the genotype at the QTL \cite{Gorkin et al. 2019 (Common DNA sequence variation influences 3-dimensional conformation of the human genome)}. The significance of such studies will only grow as we continue to appreciate the full extent of genomic structural variation between human individuals \cite{Collins et al. 2020 (A structural variation reference for medical and population genetics)}. In conclusion, I am proud to have been a part of this research community during my PhD, and know that, as the 3D genome field evolves and technology continues to improve, it will help fulfill a central promise of human genetics: to understand the connection between genetic variation and phenotypic variation. 